\documentclass[11pt]{article}
\usepackage[a4paper,margin=1in]{geometry}
\usepackage{graphicx}  
\usepackage{amsmath,amssymb}
\usepackage{bm}         
\usepackage{hyperref}   
\usepackage{mathtools}  
\usepackage{siunitx}   
\usepackage{caption}
\usepackage{booktabs}
\usepackage{listings}   % code listings
\usepackage{xcolor}

% -----------------------------
% Listings (Python) style
% -----------------------------
\definecolor{codebg}{RGB}{248,248,248}
\definecolor{codeframe}{RGB}{220,220,220}
\lstdefinestyle{py}{
  language=Python,
  backgroundcolor=\color{codebg},
  basicstyle=\ttfamily\small,
  frame=single,
  framerule=0.3pt,
  rulecolor=\color{codeframe},
  showstringspaces=false,
  tabsize=2,
  keywordstyle=\color{blue!60!black}\bfseries,
  commentstyle=\color{green!40!black}\itshape,
  stringstyle=\color{red!60!black}
}

% -----------------------------
% Title
% -----------------------------
\title{\includegraphics[height=2.5cm]{Strath-logo.jpg}\\Computational Methods Assignment 2: No.\ 2}
\author{Omondi EdJoel, 095892}
\date{September 2025}

\begin{document}
\maketitle

\section{Question 2}
\subsection{Generalized Black--Scholes PDE with Stochastic Volatility (Derivation)}

\paragraph{Model.}
Let the underlying and its (instantaneous) variance evolve as
\begin{align*}
    dS_t &= \mu\,S_t\,dt + \sigma_t\,S_t\,dZ_t, 
    \qquad v_t := \sigma_t^{2},\\
    dv_t &= \kappa(\theta - v_t)\,dt + \gamma \sqrt{v_t}\,dW_t,
\end{align*}
with correlated Brownian motions $Z,W$ such that
\[
\mathrm{Cov}(dZ_t,dW_t)=\rho\,dt, \qquad |\rho|\le 1.
\]
Let $U=U(S,v,t)$ denote the time-$t$ value of a European-style claim maturing at $T$.

\paragraph{It\^o's formula for $U(S,v,t)$.}
The quadratic (co)variations are
\[
d\langle S\rangle_t = \sigma_t^2 S_t^2\,dt = v_t S_t^2\,dt,\qquad
d\langle v\rangle_t = \gamma^2 v_t\,dt,\qquad
d\langle S,v\rangle_t = \mathrm{Cov}(dS_t,dv_t) = \rho\,\gamma\,v_t S_t\,dt.
\]
Applying It\^o to $U(S_t,v_t,t)$,
\begin{align*}
dU_t
&= U_t\,dt + U_S\,dS_t + U_v\,dv_t
   + \tfrac12 U_{SS}\,d\langle S\rangle_t
   + \tfrac12 U_{vv}\,d\langle v\rangle_t
   + U_{Sv}\,d\langle S,v\rangle_t\\[4pt]
&= \Big(
      U_t
    + \mu S\,U_S
    + \kappa(\theta - v)\,U_v
    + \tfrac12 v S^2\,U_{SS}
    + \rho\,\gamma\,v S\,U_{Sv}
    + \tfrac12 \gamma^2 v\,U_{vv}
   \Big) dt
   \\
&\quad
 + U_S\,\sigma S\,dZ_t
 + U_v\,\gamma \sqrt{v}\,dW_t,
\end{align*}
where, for brevity, we write $S=S_t$, $v=v_t$, $\sigma=\sigma_t$, and $U_t=\partial U/\partial t$, $U_S=\partial U/\partial S$, etc.

\paragraph{Risk-neutral valuation and volatility risk premium.}
In incomplete stochastic-volatility markets, the variance factor $v$ carries an
(unspanned) risk. Introduce a (possibly affine) market price of volatility risk
$\lambda(v)$; a common linear specification is $\lambda(v)=\lambda v$.
Under the risk-neutral measure $Q$:
\[
\frac{dS_t}{S_t} = r\,dt + \sigma_t\,dZ_t^{\,Q},\qquad
dv_t = \big[\kappa(\theta - v_t) - \lambda(v_t)\big]\,dt + \gamma \sqrt{v_t}\,dW_t^{\,Q},
\]
with $\mathrm{Cov}(dZ_t^{\,Q},dW_t^{\,Q})=\rho\,dt$. The $Q$-generator $\mathcal L^Q$ acting on $U$ is then
\[
\mathcal L^Q U
= r S\,U_S
+ \big[\kappa(\theta - v) - \lambda(v)\big]\,U_v
+ \tfrac12 v S^2\,U_{SS}
+ \rho\,\gamma\,v S\,U_{Sv}
+ \tfrac12 \gamma^2 v\,U_{vv}.
\]

\paragraph{Pricing PDE (no-arbitrage / Feynman--Kac).}
Discounted claim values are $Q$-martingales; equivalently, $U$ solves
\[
U_t + \mathcal L^Q U - r U = 0.
\]
With $\lambda(v)=\lambda v$ this becomes the \emph{generalized Black--Scholes PDE}:
\[
\boxed{
\frac{1}{2}\,v S^{2}\,\frac{\partial^{2}U}{\partial S^{2}}
\;+\;
\rho\,\gamma\,v S\,\frac{\partial^{2}U}{\partial v\,\partial S}
\;+\;
\frac{1}{2}\,\gamma^{2} v\,\frac{\partial^{2}U}{\partial v^{2}}
\;+\;
r S\,\frac{\partial U}{\partial S}
\;+\;
\big[\kappa(\theta - v) - \lambda v\big]\,\frac{\partial U}{\partial v}
\;-\; r U
\;+\;
\frac{\partial U}{\partial t}
\;=\; 0.
}
\]

\paragraph{Terminal/Boundary conditions.}
For a European payoff $\Phi(S_T)$ one imposes
$U(S,v,T)=\Phi(S)$ together with appropriate growth and boundary conditions in $(S,v)$
to ensure uniqueness.

\bigskip
\hrule
\bigskip

\appendix
\section*{Appendix: Starter Code (Simulation of $(S_t,v_t)$)}

\noindent
\textbf{Purpose.} The below shows Python code to simulate correlated $(S_t,v_t)$ over $[0,T]$ under the physical measure (or $Q$ if you set $\mu=r$ and replace the $v$-drift with $\kappa(\theta-v)-\lambda v$). This is \emph{not} needed for the PDE derivation, but handy if you want to \emph{illustrate} the model dynamics alongside your write-up.

\begin{lstlisting}[style=py,caption={Stochastic volatility (CIR-type variance) with correlated Brownian motions}]
import numpy as np

def simulate_sv_paths(
    S0=100.0, v0=0.04, mu=0.05, r=0.05,
    kappa=1.5, theta=0.04, gamma=0.5,
    rho=-0.6, T=1.0, steps=1000, paths=10000, seed=42,
    risk_neutral=False, lamb=0.0
):
    """
    Log-Euler for S (keeps S>0), full truncation Euler for v (keeps v >= 0).
    If risk_neutral=True, use mu=r and v-drift kappa(theta-v)-lambda*v.
    """
    rng = np.random.default_rng(seed)
    dt  = T / steps
    S   = np.full(paths, S0, dtype=float)
    v   = np.full(paths, v0, dtype=float)

    for _ in range(steps):
        Z1 = rng.standard_normal(paths)
        Z2 = rng.standard_normal(paths)
        dW1 = np.sqrt(dt) * Z1
        dW2 = np.sqrt(dt) * (rho * Z1 + np.sqrt(max(1 - rho**2, 0.0)) * Z2)

        vpos = np.maximum(v, 0.0)
        # choose drift for v
        if risk_neutral:
            drift_v = kappa * (theta - vpos) - lamb * vpos
            drift_S = r
        else:
            drift_v = kappa * (theta - vpos)
            drift_S = mu

        # log-Euler for S_t:
        S *= np.exp((drift_S - 0.5 * vpos) * dt + np.sqrt(vpos) * dW1)
        # full truncation Euler for v_t:
        v += drift_v * dt + gamma * np.sqrt(vpos) * dW2

    return S, np.maximum(v, 0.0)

# Example (test):
if __name__ == "__main__":
    ST, vT = simulate_sv_paths(
        S0=100, v0=0.04, mu=0.05, r=0.05,
        kappa=1.0, theta=0.04, gamma=0.5,
        rho=-0.5, T=1.0, steps=4000, paths=20000,
        risk_neutral=True, lamb=0.0, seed=123
    )
    print("ST mean:", ST.mean(), "  vT mean:", vT.mean())
\end{lstlisting}

\end{document}
